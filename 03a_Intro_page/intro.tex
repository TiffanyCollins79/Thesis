\chapter{Introduction}

The Aurora Borealis (aka the Northern lights) have fascinated humans as far back as the Stone Ages and have been integrated into the mythology of many cultures (see \cite{1980mlas.book.....E} and references within). These lights (and the Southern lights) are the result of charged particles emitted by solar flares that are channelled by the Earth's magnetic field to the poles. When they collide with the Earth's atmosphere, they excite oxygen and nitrogen atoms that then decay back into their ground state to produce dazzling green and red light. Charged particles from the sun have energies up to $10^{10}~\eV$ and are only the tip of the iceberg \citep{2011hea..book.....L}. 
\par~\par
The Earth's atmosphere is constantly bombarded by charged particles known as cosmic rays (CRs) that can have energies up to $10^{20}~\eV$ \citep{alma9924446790001811}. These cosmic rays originate from outside the solar system from Galactic sources such as pulsar wind nebulae (PWNe), supernova remnants (SNRs) and stellar clusters or extra-Galactic sources like Galaxy clusters and active Galactic nuclei.
\par~\par
Cosmic rays must then traverse through extreme environments in the Milky Way to reach Earth. On their journey, cosmic rays interact with magnetic fields or collide with gas and the cosmic microwave background to release electromagnetic energy from X-rays to gamma rays. Gamma rays convey information about their cosmic-ray predecessors such as where cosmic rays are accelerated, their energy and how they propagate from their source. Essentially, gamma rays are a highly accessible tracer or `smoking gun' of cosmic rays! 
\par~\par 
While our ancestors used the naked eye to observe celestial bodies, modern day astronomers have access to instruments that can observe the highest energy gamma rays. The cutting-edge High Energy Stereoscopic System (H.E.S.S.) is one such observatory that has the capability of observing gamma rays with energies up to $100~\TeV$ \citep{HESS}. In 2018, the H.E.S.S. collaboration released their second Galactic plane survey which consisted of 78 $\TeV$ gamma-ray sources \citep{2018A&A...612A...1H}. Of these sources, 20 are confirmed $\TeV$ pulsar wind nebulae and a further 36 are possible PWNe candidates. This makes pulsar wind nebulae the most numerous class of the Galactic $\TeV$ gamma-ray sources.
\par~\par 
Supernovae occur when a massive star core collapses, or when a white dwarf `re-ignites' and triggers runaway nuclear fusion. The supernova as well as gravitational collapse can compress the core of massive stars past the density of a star to \textcolor{purple}{the density inside the nucleus of an atom}. This rapidly rotating compact object, known as a neutron star, emits two beams of electromagnetic radiation from its magnetic poles. \textcolor{purple}{The magnetic and rotation axes of the neutron star do not necessarily align, leading to the beams being observed as a} series of pulses when the direction of the beam points towards Earth. \textcolor{purple}{In this scenario, the neutron star is classified as a pulsar.} The high magnetic fields of the pulsar can strip cosmic rays from its surface to form powerful winds known as pulsar wind nebula. PWNe have been observed across the electromagnetic spectrum, from low frequency radio waves \citep{1968Natur.217..709H} up to high-energy $\PeV$ gamma rays \citep{doi:10.1126/science.abg5137}.
\par~\par
Cosmic rays escaping the PWN experience diffusion (where cosmic rays scatter off magnetic fields and ISM gas, resulting in its overall motion being described by a random walk) and/or advection (an overall bulk motion of cosmic rays in a certain direction). It has been proposed that advection dominates particle transport close to the pulsar while diffusion dominates the outer reaches of the PWN \citep{2020A&A...636A.113G, 2021PhRvD.104l3017R}. Additionally, cosmic rays interact with their environment to radiate photons at a rate related to their energy. The combination of diffusion, advection and radiative processes will influence the morphology and spectral information of the PWN.
\par~\par 
This thesis will model the multi-wavelength emission towards the $\TeV$ PWN \mbox{HESS\,J1825-137} by combining cosmic ray transport theory and observations from state-of-the-art instruments such as H.E.S.S.. \mbox{HESS\,J1825-137} is one of the brightest $\TeV$ PWN with extended $\GeV$ emission $\approx \ang{2.5}$ to the south \citep{2019MNRAS.485.1001A}. This makes \mbox{HESS\,J1825-137} an ideal laboratory to study the relativistic transportation of cosmic rays towards PWN. By modelling the emission towards \mbox{HESS\,J1825-137}, insight into the cosmic-ray transportation will be gained.
\par~\par
The thesis will be structured as following:
\autoref{sec:01_PWN_chapter} will provide an overview in $\TeV$ Pulsar Wind Nebulae, cosmic rays and the processes in which cosmic rays convert their energy into gamma rays. \autoref{sec:02_astronomy} describes some of the instruments whose data products were used in this thesis and their observational techniques. \autoref{06_ISM} discusses interstellar gas, its implications in this thesis and how the gas can be detected. \autoref{sec:07_particle_ev} will delve deeper into cosmic-ray propagation how their energy distribution evolves in time for a simple region of interstellar gas. The techniques discussed in \autoref{sec:07_particle_ev} will be applied to the $\GeV$ gamma-ray emission to south of \mbox{HESS\,J1825-137} in order to gain insight into underlying particle acceleration and transportation of this region. This original work was published in the peer reviewed journal Monthly Notices of the Royal Astronomical Society (MNRAS) and can be viewed in \autoref{sec:08}. \autoref{sec:09_multizone} will expand on the cosmic-ray propagation theory discussed in \autoref{sec:07_particle_ev} by modelling the transport of CRs in complex regions of ISM with varying magnetic fields, gas distribution and diffusion rates. \autoref{sec:10_second_paper} applies this model to \mbox{HESS\,J1825-137} in order to explain the extended multi-wavelength emission towards \mbox{HESS\,J1825-137} and to constrain the gamma-ray contamination of nearby unidentified $\TeV$ object \mbox{HESS\,J1826-130} by \mbox{HESS\,J1825-137}. \autoref{sec:10_summary} will summarise the work conducted in this thesis and discuss any future work.

\section*{Acronyms and Abbreviations}

\gloss{Bremsstrahlung}{brem}
\gloss{CMB}{cosmic microwave background}
\gloss{CTA}{Cherenkov Telescope Array}
\gloss{FIR}{far-infrared}
\gloss{Fermi-LAT}{\textit{Fermi} Large Area Telescope}
\gloss{GC}{Galactic centre}
\gloss{HAWC}{High Altitude Water Cherenkov Experiment}
\gloss{H.E.S.S.}{High Energy Sterescopy System}
\gloss{HGPS}{HESS Galactic Plane Survey}
\gloss{IC}{inverse Compton}
\gloss{IR}{infrared}
\gloss{ISM}{interstellar medium}
\gloss{LSR}{local standard of rest}
\gloss{LHAASO}{Large High Altitude Air Shower Observatory}
\gloss{NIR}{near-infrared}
\gloss{p-p}{proton-proton}
\gloss{PWN}{pulsar wind nebula}
\gloss{SED}{Spectral Energy Distribution}
\gloss{SNR}{supernova remnant}
\gloss{sync}{synchrotron}
\gloss{UV}{Ultra Violet}
\gloss{VHE}{Very High Energy}