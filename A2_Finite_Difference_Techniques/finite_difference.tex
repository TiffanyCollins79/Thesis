\chapter{Finite Difference Techniques applied to the Transport Equation} \label{sec:A2_finite_difference}

Often, these differential equations cannot be solved analytically unless in certain scenarios. For example, the transport equation (see \autoref{eq:chapter_9_full_numerical_diffusion_equation}) can only be solved for a impulsive source of cosmic rays. Finite difference techniques are utilised to solve differential equations numerically by approximating derivatives by finite differences. This section will describe the basics of finite difference techniques and the application to the transport equation.

\section{Diffusion} \label{sec:A2_diffusion}

The diffusive component of the transport equation (see \autoref{eq:chapter_9_full_numerical_diffusion_equation}) is described by:

\begin{equation}
    \begin{aligned}
    \pdv{n\qty(\boldsymbol{r},\gamma,t)}{t}&=\nabla \qty(\overline{\overline{D}}\cdot\nabla n\qty(\boldsymbol{r},\gamma,t))\text{ .} 
    \end{aligned} \label{eq:A2_finite_difference_diffusion}
\end{equation}
\noindent A region of interstellar gas can be split into a 3D grid of voxels. Each voxel has width $\Delta x$, $\Delta y$ and $\Delta z$zs. Assuming that diffusion is isotropic ($\overline{\overline{D}}=D$), \autoref{eq:A2_finite_difference_diffusion} at position $x$, $y$, $z$ and time $t$ becomes:

\begin{equation}
    \begin{aligned}
    \pdv{n\qty(\boldsymbol{r},\gamma,t)}{t}\bigg|_{i}^t &= \nabla D\cdot\nabla n + D\nabla^2 n \\
    &= \sum_{i} \pdv{D}{i}\pdv{n}{i}+ D_i\pdv[2]{n}{i}\text{ ,} 
    \end{aligned} \label{eq:A2_finite_difference_diffusion_expanded}
\end{equation}
\noindent where $i$ represents the three cartesian axes $x$, $y$ and $z$. Assuming that diffusion is spatially independent, $\pdv{D}{i}=0$, the Taylor series expansion of $n$ around point i becomes:

\begin{subequations}
    \begin{alignat}{1}
    n_{i+\Delta i} &=n_{i}+\Delta i \pdv{n}{i}\bigg|_{i+\Delta i} + \frac{\Delta i^2}{2}\pdv[2]{n}{i}\bigg|_{i+\Delta i}+ \frac{\Delta i^3}{3!}\pdv[3]{n}{i}\bigg|_{i+\Delta i} +\cdot\cdot\cdot \label{eq:A2_taylor_series_plus} \\
    n_{i-\Delta i} &=n_{i}-\Delta i \pdv{n}{i}\bigg|_{i-\Delta i} + \frac{\Delta i^2}{2}\pdv[2]{n}{i}\bigg|_{i-\Delta i}- \frac{\Delta i^3}{3!}\pdv[3]{n}{i}\bigg|_{i-\Delta i} +\cdot\cdot\cdot\text{ .}   \label{eq:A2_taylor_series_neg}
    \end{alignat} \label{eq:A2_taylor_series}
\end{subequations} 
\noindent Assuming $\Delta i\ll 1$, \autoref{eq:A2_taylor_series} becomes:

\begin{subequations}
    \begin{alignat}{1}
    n_{i+\Delta i} &=n_{i}+\Delta i \pdv{n}{i}\bigg|_{i+\Delta i} + O\qty(\Delta i^2) \label{eq:A2_taylor_series_plus2} \\
    n_{i-\Delta i} &=n_{i}-\Delta i \pdv{n}{i}\bigg|_{i-\Delta i} + O\qty(\Delta i^2)\text{ ,}  \label{eq:A2_taylor_series_neg2}
    \end{alignat} \label{eq:A2_taylor_series2}
\end{subequations} 
\noindent where $O\qty(\Delta i^3)$ is a truncation error due to higher orders. Rearranging \autoref{eq:A2_taylor_series2} to first order gives:

\begin{subequations}
    \begin{alignat}{1}
    \pdv{n}{i} \bigg|_{i+\Delta i} &\approx\frac{n_{i+\Delta i}-n_i}{\Delta i} \label{eq:A2_forward_tecnique} \\
    \pdv{n}{i} \bigg|_{i-\Delta i} &\approx\frac{n_i-n_{i-\Delta i}}{\Delta i}\text{ ,}  \label{eq:A2_backward_tecnique}
    \end{alignat}
\end{subequations}

\noindent where \autoref{eq:A2_forward_tecnique} and \autoref{eq:A2_backward_tecnique} are the forward and backward difference respectively. The central difference can be found by combining \autoref{eq:A2_forward_tecnique} and \autoref{eq:A2_backward_tecnique}:

\begin{equation}
    \begin{aligned}
    \pdv{n}{i} \bigg|_{i+\Delta i}+\pdv{n}{i} \bigg|_{i-\Delta i}&=\frac{n_{i+\Delta i}-n_i}{\Delta i} +\frac{n_i-n_{i-\Delta i}}{\Delta i} \\
    \therefore \pdv{n}{i} \bigg|_i &=\frac{n_{i+\Delta i}-n_{i-\Delta i}}{\Delta i}\text{ .} 
    \end{aligned}
\end{equation}
\noindent The second derivative of $n$ can be found by combining \autoref{eq:A2_taylor_series_plus} and \autoref{eq:A2_taylor_series_neg} up to the second order:

\begin{equation}
    \begin{aligned}
        n_{i+\Delta i}+n_{i-\Delta i} &=2n_{i} + 2\frac{\Delta i^2}{2}\pdv[2]{n}{i}\bigg|_{i} \\
        \pdv[2]{n}{i} \bigg|_{i}&=\frac{n_{i+\Delta i}+n_{i-\Delta i}-2n_i}{\Delta i^2}\text{ .} 
    \end{aligned}
\end{equation}
\noindent Applying the forward difference to the LHS of \autoref{eq:A2_finite_difference_diffusion_expanded} and central difference to the RHS of \autoref{eq:A2_finite_difference_diffusion_expanded} for time step $\Delta t$:

\begin{equation}
    \begin{aligned}
    \frac{n_i^{t+\Delta t}-n_i^t}{\Delta t}&=\sum_{i=x,y,z}\frac{D_{i+\Delta i}-D_{i-\Delta i}}{2\Delta i}\cdot \frac{n_{i+\Delta i}^t-n_{i-\Delta i}^t}{2\Delta i}+D_i\frac{n_{i+\Delta i}^t+n_{i-\Delta i}^t-2n_i^t}{\Delta i^2}\text{ .} 
    \end{aligned}
\end{equation}
\noindent For isotropic diffusion ($\nabla D=0$, i.e. $D_{i+\Delta i}=D_{i-\Delta i}$):

\begin{equation}
    \begin{aligned}
    n_i^{t+\Delta t}&=n_i^t+\sum_{i=x,y,z}\frac{\Delta t}{\Delta i^2}D_i\qty(n_{i+\Delta i}^t+n_{i-\Delta i}^t-2n_i^t) \\
    &=n_i^t+\sum_{i=x,y,z} \frac{\Delta t}{\Delta i^2}\qty[D_i\qty(n_{i+\Delta i}^t-n_i^t)+D_i\qty(n_{i-\Delta i}^t-n_i^t)]\text{ .} 
    \end{aligned} \label{eq:A2_diffusion_finite_diff}
\end{equation}
\noindent i.e. \autoref{eq:A2_diffusion_finite_diff} is the solution of \autoref{eq:A2_finite_difference_diffusion} at time $t+\Delta t$ using finite difference techniques.

\subsection{Stability of the Finite Difference} \label{sec:A2_finite_stability}

Von Neumann stability analysis (or Fourier stability analysis) can be used to check the numerical stability of finite difference techniques \citep{alma9929637001811}. For a general solution:
\begin{equation}
    \begin{aligned}
        U\qty(t+\Delta t, i)&=\sum_{m=-n}^n C_m U\qty(t,i+m\Delta i)\text{ ,} 
    \end{aligned} \label{eq:A3_von_newmann_example}
\end{equation}
\noindent where there are $2n+1$ points to calculate $U\qty(t+\Delta t, i)$, the error, $\epsilon_m^t$, with respect to the analytical solution is:

\begin{equation}
    \begin{aligned}
        \epsilon_m^t=U_m^t-u_m^t\text{ ,} 
    \end{aligned} \label{eq:A2_error_function}
\end{equation}
\noindent with $u_m^t$ being the analytical solution at point $m,t$. Then, $\epsilon_m$ satisfies \autoref{eq:A3_von_newmann_example}:

\begin{equation}
    \begin{aligned}
        \epsilon \qty(t+\Delta t, i)&=\sum_{m=-n}^n C_m \epsilon\qty(t,i+m\Delta i)\text{ ,} 
    \end{aligned} \label{eq:A3_von_newmann_example}
\end{equation}
\noindent For linear differential equations with periodic boundary conditions, the error can be treated as a time-dependent Fourier expansion with wave number $k$ \citep{cdi_wiley_ebooks_10_1002_9783527683147_ch11_ch11}.:

\begin{equation}
    \begin{aligned}
    \epsilon\qty(x,t)&=A\qty(t)e^{jki}=A^te^{jki}\text{ ,} 
    \end{aligned} \label{eq:A2_diffusion_finite_trial_sol}
\end{equation}
\noindent where $j$ is the complex number and \autoref{eq:A2_diffusion_finite_trial_sol} is the time-dependent spatial Fourier expansion with wave number $k$. For the solution to be stable:

\begin{equation}
    \begin{aligned}
        G=\bigg| \frac{A^{t+1}}{A^t}\bigg| \leq 1\text{ ,} 
    \end{aligned}
\end{equation} 
\noindent with $G$ being the amplification factor. Therefore:

\begin{equation}
    \begin{aligned}
    \epsilon_i^{t+1}&= A^{t+1}e^{jki} \\
    \epsilon_{i+1}^{t}&= A^{t}e^{jk\qty(i+\Delta i)} \\
    \epsilon_{i-1}^{t}&= A^{t}e^{jk\qty(i-\Delta i)}\text{ .} 
    \end{aligned} \label{eq:A2_diffusion_finite_trial_sol2}
\end{equation}
Applying Von Newmann stability analysis to the transport equation, \autoref{eq:A2_diffusion_finite_trial_sol} becomes:

\begin{equation}
    \begin{aligned}
    A^{t+1}e^{jki}&=A^te^{jki}+D'\qty(A^{t}e^{jk\qty(i+\Delta i)}+A^{t}e^{jk\qty(i-\Delta i)}-2A^te^{jki}) \\
    \frac{A^{t+1}}{A^t}&=1+D'\qty(e^{j\theta}+e^{-j\theta}-2) \\
    &=1+D'\qty(2\cos\theta-2) \\
    &=1-4D'\sin^2\frac{\theta}{2}\text{ ,} 
    \end{aligned}
\end{equation}
\noindent where $D'=D\Delta t/\Delta i^2$ and $\theta=k\Delta i$. For the solution to be stable, $\lvert \frac{A^{t+1}}{A^t}\rvert\leq 1$. Hence:

\begin{equation}
    \begin{aligned}
    1-4D'\sin^2\frac{k\Delta i}{2} &\geq -1 \\
    \text{\& } 1-4D'\sin^2\frac{k\Delta i}{2} &\leq 1 \text{ .} 
    \end{aligned} \label{eq:A2_diffusion_finite_stability_cond}
\end{equation}
\noindent As $0\leq \sin^2\frac{k\Delta i}{2}\leq 1$, \autoref{eq:A2_diffusion_finite_diff} is stable when:

\begin{equation}
    \begin{aligned}
    D'=D\frac{\Delta t}{\Delta i^2}&\leq \frac{1}{2} \\
    D&\leq \frac{1}{2}\frac{\Delta i^2}{\Delta t}\text{ .} 
    \end{aligned}
\end{equation}
\noindent i.e. finite difference techniques can be applied to the transport equation when the diffusion coefficient, $D$, is less than or equal to half of the `maximum' diffusion coefficient implied by the relation $\Delta i^2/\Delta t$.

\section{Advection}

The advective component of the transport equation is given by:

\begin{equation}
    \begin{aligned}
        \pdv{n\qty(\boldsymbol{r},\gamma,t)}{t}&=-\nabla\qty(n\boldsymbol{v_a})\text{ ,} 
    \end{aligned}
\end{equation}
\noindent where $v_a$ is the advective velocity of cosmic rays. Assuming $v_a$ is spatially independent:

\begin{equation}
    \begin{aligned}
    \pdv{n\qty(\boldsymbol{r},\gamma,t)}{t}&=-\boldsymbol{v_a}\nabla n  \\
    \pdv{n}{t}\bigg|_{i}^t&=-v_i\pdv{n}{i}\text{ ,} 
    \end{aligned} \label{eq:A2_advection_equation}
\end{equation}
\noindent where $i$ considers the three Cartesian axes $x$, $y$ and $z$. Applying the forward difference to the LHS of \autoref{eq:A2_advection_equation} and the central difference to the RHS:

\begin{equation}
    \begin{aligned}
    \frac{n^{t+\Delta t}_i-n^t_i}{\Delta t}&=-\sum_{i=x,y,z} v_i\frac{n_{i+\Delta i}^t-n_{i-\Delta i}^t}{2\Delta i} \\
    n_i^{t+\Delta t}&=n_i^t-\frac{v_i}{2}\cdot\frac{\Delta t}{\Delta x}\qty(n_{i+\Delta i}^2-n_{i+\Delta i}^2) \\
    &=n_i^t-\frac{v'}{2}\qty(n_{i+\Delta i}^t-n_{i+\Delta i}^t)\text{ ,} 
    \end{aligned} \label{eq:A2_advection_central_diff}
\end{equation}
\noindent where $v'=\frac{\Delta t}{\Delta x}$. Applying the Von Neumann stability analysis to \autoref{eq:A2_advection_central_diff}:

\begin{equation}
    \begin{aligned}
    A^{t+1}e^{ikx}&=A^te^{ikx}+\frac{v'}{2}\qty(e^{ikx}e^{ik\Delta x}-e^{ikx}e^{-ik\Delta x}) \\
    \frac{A^{t+1}}{A^t}&=1-\frac{v'}{2}\qty(e^{i\theta}-e^{-i\theta}) \\
    &=1-iv'\sin\theta\text{ .} 
    \end{aligned}
\end{equation}
\noindent As $\lvert \frac{A^{t+1}}{A^t}\rvert\leq 1$:

\begin{equation}
    \begin{aligned}
        \bigg| \frac{A^{t+1}}{A^t}\bigg|=1+v'^2\sin^2\theta &\leq1\text{ .} 
    \end{aligned}
\end{equation}
\noindent Therefore the numerical solution of advection that utilises a central finite difference is only stable when:

\begin{equation}
    \begin{aligned}
        \sin^2\theta=\sin^2\qty(k\Delta x)\leq 0\text{ .} 
    \end{aligned}
\end{equation}
\noindent i.e. the solution is only stable when the error function (see \autoref{eq:A2_error_function}) has wave number:

\begin{equation}
    \begin{aligned}
        k\Delta x = n\pi\text{ ,} 
    \end{aligned}
\end{equation}
\noindent where $n=0$, $1$, $\cdot\cdot\cdot$. Therefore, the numerical solution of advection that utilises central finite difference methods is unstable. Instead, the forward and backward difference in spatial coordinates can be applied to \autoref{eq:A2_advection_equation}:
\begin{subequations}
    \begin{alignat}{2}
    n_i^{t+\Delta t}&=n_i^t-v'\qty(n_{i+\Delta i}^t-n_{i}^t)\text{,}&\quad\text{Forward Difference} \\
    n_i^{t+\Delta t}&=n_i^t-v'\qty(n_{i}^t-n_{i-\Delta i}^t)\text{,}&\quad\text{Backward Difference}\text{ .} 
    \end{alignat} \label{sec:A2_advection}
\end{subequations}
\noindent Applying the Von Neumann analysis for the forward difference:

\begin{equation}
    \begin{aligned}
    \frac{A^{t+1}}{A^t}&=1-v'\qty(e^{i\theta}-1) \\
    &=1-v'\qty(\cos\theta+i\sin\theta-1) \\
    &=1+v'\qty(1-\cos\theta)+iv'\sin\theta\text{ ,} 
    \end{aligned}
\end{equation}
\noindent and the backward difference:

\begin{equation}
    \begin{aligned}
    \frac{A^{t+1}}{A^t}&=1-v'\qty(1-e^{-i\theta}) \\
    &=1-v'\qty(1-\cos\theta+i\sin\theta) \\
    &=1-v'\qty(1-\cos\theta)-iv'\sin\theta\text{ .} 
    \end{aligned}
\end{equation}
\noindent The forward and backward difference can be summarised as:

\begin{equation}
    \begin{aligned}
    \frac{A^{t+1}}{A^t}&=1\pm v'\qty(1-\cos\theta)\pm iv'\sin\theta\text{ ,} 
    \end{aligned}
\end{equation}
\noindent Therefore:

\begin{equation}
    \begin{aligned}
    \bigg| \frac{A^{t+1}}{A^t} \bigg|^2&=1+v'^2+v'^2\cos^2\theta - 2v'^2\cos\theta \pm 2c \mp 2v'\cos\theta + v'^2\sin^2\theta \\
    &=1-2v'\qty(v'\cos\theta -v'\pm \cos\theta \mp 1) \\
    &=1\mp 2v'\qty(1\mp v')\qty(1-\cos\theta)\text{ .} 
    \end{aligned}
\end{equation}
\noindent For the finite difference to be stable:

\begin{equation}
    \begin{aligned}
    1&\geq \bigg| \frac{A^{t+1}}{A^t} \bigg|\\
    1&\geq\bigg| \frac{A^{t+1}}{A^t} \bigg|^2\\
    0&\geq \mp 2v'\qty(1\mp v')\qty(1-\cos\theta)\text{ .} 
    \end{aligned}
\end{equation}
\noindent As $\qty(1-\cos\theta)\geq 0$:

\begin{equation}
    \begin{aligned}
    1&\geq \pm v\frac{\Delta t}{\Delta x}\text{ .} 
    \end{aligned} \label{eq:A3_advection_stability}
\end{equation}
\noindent i.e. \autoref{eq:A3_advection_stability} is stable when a particle has speed less or equal to the `maximum' speed ($\Delta x/\Delta t$) implied by the voxel size and time step. When $v\leq 0$, only the forward difference is stable. Similarly, when $v\geq 0$, only the backward difference is stable.