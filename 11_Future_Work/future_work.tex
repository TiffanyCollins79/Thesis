\chapter{Conclusion and Future Work} \label{sec:10_summary}

This thesis investigated the origin of the X-ray to high-energy gamma-ray emission towards the extended $\TeV$ Pulsar Wind Nebula (PWNe) \mbox{HESS\,J1825-137} and adjacent source \mbox{HESS\,J1826-130}. PWN constitute the largest source class in the H.E.S.S. galactic plane survey (see \autoref{sec:02_HESS}) and have been identified as possible PeVatron source candidates; a source capable of accelerate cosmic rays (protons and electrons) greater than $1~\PeV=10^{15}~\eV$ (see \autoref{sec:chapter_1_PeVatrons}). The radio to gamma-ray emission towards \mbox{HESS\,J1825-137} was modelled by linking cosmic-ray transport theory and radiative losses with ISM data towards \mbox{HESS\,J1825-137}.
\par~\par
This thesis first investigated the extended $\GeV$ emission towards the Galactic-south of \mbox{HESS\,J1825-137} (named GeV-ABC) as revealed by \cite{2019MNRAS.485.1001A}. The $\GeV$ emission was subdivided into three regions (GeV-A, GeV-B and GeV-C) based on the three peaks observed in the TS map by \cite{2019MNRAS.485.1001A}. Using CO(1-0) data from the Nanten2 radio telescope, an analysis of the ISM towards GeV-ABC was undertaken in order to identify any association of gas clouds with the $\GeV$ emission. It was noted that a dense cloud of gas ($n\approx 80~\centimeterminusthree$) correlated with the physical position of GeV-B in the velocity range $15-30~\kmpersec$ ($1.6-2.8~\kpc$), coinciding with the distance estimate to \mbox{LS\,5039} of $2.5~\kpc$. The dense cloud also spatially correlated with a region of reduced H$\alpha$ emission and the H$\alpha$ rims detected by \cite{2008MNRAS.390.1037S} and \cite{2016MNRAS.458.2813V} that are consistent with the radius of the progenitor SNR of \mbox{HESS\,J1825-137} ($r_\text{SNR}\approx 120~\pc$).
\par~\par
The radio to gamma-ray emission towards \mbox{GeV-ABC} was then modelled using a single-zone determine the origin of protons or electrons within this region (see \autoref{sec:08}). This study investigated whether these cosmic rays originated from the PWN and/or the progenitor SNR associated with \mbox{HESS\,J1825-137}. It found that the powering pulsar (\mbox{PSR\,J1826-1334}) must inject more than {$10^{37}~\ergspersec$} of electrons or more than $10^{39}~\ergspersec$ of protons to power \mbox{GeV-ABC}, compared to the pulsar spin-down power of $10^{36}~\ergspersec$. \textcolor{blue}{Furthermore, the distance between the pulsar and GeV-ABC ($\approx 70~\pc$) suggest that only electrons with energy greater than $10~\TeV$ are able to reach the region within the age of the system.} Similarly, the progenitor SNR must inject $>10^{51}~\ergs$ of protons, compared to the canonical SNR cosmic-ray energy budget of $10^{50}~\ergs$. Unless the extended $\GeV$ emission reflects an earlier, more powerful epoch of the PWN or the progenitor SNR is a hypernova (with a cosmic-ray energy- budget of $10^{51}~\ergs$), an accelerator associated with \mbox{HESS\,J1825-137} is unlikely to be sole origin of protons or electrons towards \mbox{GeV-ABC}.
\par~\par 
It was also postulated that the nearby binary system \mbox{LS\,5039} - either accretion onto the associated compact object or its progenitor SNR - could result in the protons or that power \mbox{GeV-ABC} (see \autoref{sec:08}). It was ascertained through single-zone modelling that to power GeV-ABC, \mbox{LS\,5039} required an injection luminosity greater than the average accretion luminosity ($\approx 10^{35}~\ergspersec$) or a minimum SNR energy budget of $10^{51}~\ergs$. Moreover, the estimated age of \mbox{LS-5039} ($\approx 0.1-1~\si{Myr}$) suggests that the SNR would be merging with the ISM. Therefore, an accelerator associated with \mbox{LS\,5039} by itself is unlikely to be the sole origin of protons or electrons responsible for GeV-ABC. A combination of an accelerator linked to \mbox{HESS\,J1825-137} and \mbox{LS\,5039} could explain the $\GeV$ emission.
\par~\par
The next part of this thesis modelled a 3D distribution of electrons towards \mbox{HESS\,J1825-137} (see \autoref{sec:09_multizone}) in order to predict the multiwavelength emission seen towards the PWN and to investigate the gamma-ray contamination of the nearby northern $\TeV$ source \mbox{HESS\,J1826-130}. This was achieved by numerically solving the cosmic-ray transport/loss equation over a 3D Cartesian grid of spatially-dependent number density and magnetic field using finite difference techniques (see \autoref{sec:09_multizone}). The \cite{2019A&A...621A.116H} suggested that both diffusive transport and advection towards lower Galactic longitudes was required to explain the asymmetric $\TeV$ morphology towards \mbox{HESS\,J1825-137}. Three different models were considered; Model 1 - isotropic diffusion, Model 2 - isotropic diffusion + advection and Model 3 - isotropic diffusion + advection + turbulent gas between \mbox{HESS\,J1825-137} and \mbox{HESS\,J1826-130} (see \autoref{sec:10_second_paper}).
\par~\par 
For a characteristic age of $21~\kiloyear$, Model 1 required an electron injection luminosity of $10^{37}~\ergspersec$ to predict the multiwavelength SED, while an older age of $40~\kiloyear$ required $10^{35}~\ergspersec$. This represents $1000\%$ and $14\%$ of the spin-down power of the pulsar respectively, suggesting that the true age of \mbox{HESS\,J1825-137} is older than what the characteristic age suggests. Both ages could not explain the asymmetric gamma-ray morphology towards \mbox{HESS\,J1825-137} and over-predicted the gamma-ray SED below $2~\TeV$ towards \mbox{HESS\,J1826-130}.
\par~\par 
The multiwavelength SED of \mbox{HESS\,J1825-137} remained essentially unchanged when Model 2 introduced an advective flow of $0.002c$ with the same input parameters as Model 1-$40~\kiloyear$. The gamma-ray morphology was investigated in detail by taking slice profiles of the predicted gamma-ray flux map towards \mbox{HESS\,J1825-137} and the uncorrelated HESS excess maps revealed by \cite{2019A&A...621A.116H}. Model 2 was able to match the shape of the gamma-ray slice profile along Galactic longitude for photons $<1~\TeV$ and greater than $10~\TeV$. However, the slice profile between $1~\TeV<E<10~\TeV$ was shallower compared to the HESS excess slice, suggesting that the parent electrons were confined within the PWN before escaping into the ISM. Notably, Model 2 still over-predicted the gamma-ray emission $<2~\TeV$ towards \mbox{HESS\,J1826-130}.
\par~\par 
Turbulent motion in the ISM results in an amplification of the magnetic field and acts as a barrier for electrons escaping into the ISM. Hence Model 3 introduced a shell of increased magnetic field gas around \mbox{HESS\,J1826-130} to replicate the observed turbulent gas between \mbox{HESS\,J1825-137} and \mbox{HESS\,J1826-130} \citep{2016MNRAS.458.2813V}. Model 3 found that an ISM shell with $B>60~\si{\micro G}$ around \mbox{HESS\,J1826-130} was required to lower the gamma-ray emission to those estimated by HESS. However, the X-ray upper limit towards \mbox{HESS\,J1286-130} constrained the shell to have a maximum strength of $20~\si{\micro G}$. This constraint violation suggests further refinement of the model is needed to fully disentangle the particle transport towards \mbox{HESS\,J1825-137} and that the turbulent ISM may play a role in the emission from \mbox{HESS\,J1826-130}.

\section*{Future Work}

Models 1, 2 and 3 in \autoref{sec:10_second_paper} made some simple assumptions of the modelled parameters. For example, the detection of $\TeV$ halos around PWN like \mbox{HESS\,J1825-137} suggest that the transport of electrons within the vicinity around the pulsar is suppressed compared to the surrounding ISM (e.g. see \cite{2018PhRvD..98f3017E}). However, the model used in \autoref{sec:10_second_paper} did not consider the spatial dependence of the diffusion coefficient. Similarly, it has been proposed that advection dominates close to the pulsar and diffusion dominates the outer reaches of the nebula (e.g. \cite{2020A&A...636A.113G, 2021PhRvD.104l3017R}), while the model here assumed isotropic diffusion with a constant 2D advective flow. Hence, future modelling towards \mbox{HESS\,J825-137} and other PWN will consider diffusion and advection to be radially dependent on the distance to the pulsar.
\par~\par
The model assumed that the position of the electron source does not change over time. Asymmetry in the progenitor SNR will result in the pulsar gaining a kick-velocity up to $300~\kmpersec$ \citep{2017ApJ...844....1K}, leading to the powering pulsar being displaced up to $\approx 6.5~\pc$ for the characteristic age of $21~\kiloyear$. This and asymmetry in the SNR reverse shock may contribute to the asymmetric gamma-ray morphology towards \mbox{HESS\,J1825-137} and will be included in the future. \textcolor{blue}{Similarly, the model assumed a time-independent magnetic field with decreasing strength from the distance to the pulsar. Due to conservation of magnetic energy density, the average magnetic field is expected to decrease over time. Hence, the magnetic field used in the model can be considered as the time-averaged magnetic field over the age of the system. Any future predictions of \mbox{HESS\,J1825-137} or similar PWN must consider a time-dependent magnetic field.}
\par~\par 
\autoref{sec:09_multizone} and \autoref{sec:10_second_paper} considered the pulsar to only be a source of electrons and neglected cosmic-ray protons and positrons. However, studies such as \cite{1992MNRAS.257..493B,10.1111/j.1745-3933.2010.00934.x,2018MNRAS.478..926O, Xin_2019, 2021ApJ...922..221L} suggest that PWNe may have an additional hadronic component which would also contribute to the secondary emission of electrons from proton-proton interactions. Future implementation will consider the co-evolution of protons, positrons and electrons and their contribution to the SED of the PWN.
\par~\par
\mbox{HESS\,J1825-137} is only one of the 20 confirmed PWNe in the H.E.S.S. Galactic Plane Survey \citep{2018A&A...612A...1H}. Clearly, the 3D model discussed in \autoref{sec:09_multizone} could be applied a population of PWNe and future work could investigate the evolution of PWNe properties (e.g. multi-wavelength morphology and SED, injection spectrum and luminosity, ect). This would provide understanding on how cosmic rays and the surrounding ISM influences the evolution of PWNe.
\par~\par
The next-generation Cherenkov Telescope Array with its improved sensitivity and angular resolution \citep{2019scta.book.....C} will significantly increase the number of detected PWNe. Features in the gamma-ray morphology around known PWN and the SED at the highest energy ($>10~\TeV$) will be resolved in finer detail compared to current gamma-ray instruments. The modelling conducted in \autoref{sec:09_multizone} and \autoref{sec:10_second_paper} could be used to predict the gamma-ray morphology towards PWNe and design future CTA observations using the python package {\tt Gammapy} \citep{gammapy:2017}.
\par~\par 
Finding, The modelling conducted in this thesis is not limited to PWNe and can be used to predict the multi-wavelength emission around other particle accelerators such as SNRs and stellar clusters.