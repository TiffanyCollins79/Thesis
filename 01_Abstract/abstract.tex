\chapter{Abstract}
Pulsars are dense, rapidly spinning remnants of massive stars. Charged particles are stripped off the surface of the pulsar by extreme magnetic fields and are accelerated beyond $\TeV$ energies, forming the powerful winds known as pulsar wind nebulae (PWNe). These particles, known as cosmic rays, escape into the interstellar medium (ISM) and interact with soft photon fields to produce gamma rays or with magnetic fields to produce radio to X-ray emission. One of the major mysteries in modern-day astrophysics is how cosmic rays escape from PWN environments.
\par~\par
\mbox{HESS\,J1825-137} is a bright, extended $\TeV$ PWN, making it an ideal laboratory to study cosmic-ray transport in PWN. Both the HAWC and LHAASO observatories have observed gamma-ray emission from \mbox{HESS\,J1ß825-137} greater than $50~\TeV$, indicative of a PeVatron; a source capable of accelerating cosmic rays up to energies greater than $10^{15}~\eV$. This thesis focuses on understanding the origin of the X-ray to gamma-ray emission towards \mbox{HESS\,J1825-137}.
\par~\par
\textit{Fermi}-LAT observations revealed extended $\GeV$ emission to the Galactic south of \mbox{HESS\,J1825-137}. The first portion of this thesis investigated whether this $\GeV$ emission originated from the PWN associated with \mbox{HESS\,J1825-137}, the progenitor supernova remnant (SNR) or a source linked to the nearby object \mbox{LS\,5039}. ISM gas analysis was first conducted towards this region in order to constrain the multi-wavelength emission. The analysis highlighted a dense cloud of CO(1-0) gas lying towards the $\GeV$ region at the same distance as \mbox{LS\,5039} that is coincident with the H$\alpha$ SNR rim associated with \mbox{HESS\,J1825-137}. The results of the gas analysis was combined with spectral energy distribution (SED) modelling to show that neither a source associated with \mbox{HESS\,J1825-137} or \mbox{LS\,5039} is likely to be the sole origin of high-energy cosmic rays towards the $\GeV$ region. A combination of both sources could result in the gamma-ray emission. This study emphasised the complexity of the region towards \mbox{HESS\,J1825-137}.
\par~\par
The second part of this thesis investigated the multi-wavelength SED and gamma-ray morphology towards \mbox{HESS\,J1825-137} to disentangle the transport mechanisms of electrons from the pulsar. Electrons escaping the PWN experience diffusion (where particles scatter off turbulence, resulting in 'random-walk') and/or advection (the bulk motion of particles). The region towards \mbox{HESS\,J1825-137} was divided into a 3D grid of ISM gas and magnetic fields based on ISM observations. The transport of electrons from the pulsar was then modelled to reproduce the multi-wavelength SED and gamma-ray morphology seen towards \mbox{HESS\,J1825-137}. A diffusive model with an advective velocity of $0.002c$ ($c$ is the speed of light) towards lower Galactic longitudes could explain observations. Additionally, a turbulent region of gas with a magnetic field between $20-60~\si{\micro G}$ is required to prevent significant gamma-ray contamination towards nearby northern $\TeV$ source, \mbox{HESS\,J1826-136}.
\par~\par
The modelling conducted in this thesis is not constrained to \mbox{HESS\,J1825-137} and can be applied to other $\TeV$ PWN or other gamma-ray sources to develop understanding of how cosmic-ray sources evolve in their respective environments.